\documentclass[11pt]{article}

\usepackage{fullpage}
\usepackage[pdftex]{graphicx}
\usepackage{setspace}
\usepackage[T1]{fontenc}
\usepackage[utf8]{inputenc}
\usepackage{hyperref}
\renewcommand{\thesection}{}
\renewcommand{\thesubsection}{\arabic{subsection}}
\makeatletter
\def\@seccntformat#1{\csname #1ignore\expandafter\endcsname\csname the#1\endcsname\quad}
\let\sectionignore\@gobbletwo
\let\latex@numberline\numberline
\def\numberline#1{\if\relax#1\relax\else\latex@numberline{#1}\fi}
\makeatother
\begin{document}

\section{\#osdev-offtopic law guide}

This document is merely a summary of the most important laws helpful for everyday activity.

\subsection{Terminology}

\begin{itemize}
\item \texttt{law}: A passed proposal. A proposal requires three contiguous votes
by unique non-bot members of the channel to be passed.

Laws need not effect active behavior on the channel, and can be passed because of
Rule of Funny.

\item \texttt{malcompliance}: The act of complying in the worst possible manner. Or,
as the Finnish define it, ``following the letter of the law while pissing on the spirit''.

Examples:
\begin{quote}
``Hey could you test sortix?'' ``test -f sortix.iso \# Yep. It's a file.'' \\
``Hey, can you give me some water?'' ``You mean aqua?'' ``...sure.'' *gives aqua fortis*
\end{quote}

\item \texttt{filibuster}: Anything that is sent on the channel, and can stop a
law from being passed, is known as a filibuster. This is line in with the literal meaning, ``obstructs progress in a legislative assembly''.

\item \texttt{proposal}: Anything that can filibuster can be a proposal.

\item \texttt{new ancient law}: A law that has always been true, 
but was only recently discovered and legislated.

\item \texttt{lawrememberer}: The person responsible for maintaining the lawlist,
currently `nortti'.

\item \texttt{lawspeaker}: The person who interprets and clarifies the law,
currently `nortti'.

\item \texttt{triminority}: The three required to pass a law. Can be used to refer to
an actual group, or a hypothetical group.

\item \texttt{triumvirate}: People who are more active with channel work, and have additional rights with ChanServ. Currently consists of `FireFly', `heddwch', `meowrobot', `nortti', `puckipedia', `shikhin', `sortie', `vehk', and `ybden'.

\item \texttt{vote}: Anything described under `Voting.Syntaxen'.
\end{itemize}

\subsection{Voting}

\subsubsection{Basics}

At every moment, there is an active proposal and a vote count.

If a vote that doesn't
refer to the current active proposal is cast, the active proposal changes to the new
proposal, and the vote count resets to 0. A filibuster sets the active proposal to itself and resets the vote count to 0.

A vote increments the vote count by 1 after change of proposal (if required).

When
the vote count reaches 3, the active proposal becomes a law.

\subsubsection{Syntaxen}
There are several different kinds of syntaxes for voting on laws. They're all based
on the original syntax of \texttt{:D}, with various modifications.

\begin{itemize}

\item \texttt{:D}

The most basic form. Votes for the current active proposal.

\item \texttt{:D\~{}N}

Votes N proposals back. Is 0-indexed, so \texttt{:D\~{}0} is
equivalent to \texttt{:D}.

\item \texttt{:D\^{} :D\^{}\^{} :D\^{}\^{}\^{} ...}

Equivalent to \texttt{:D\~{}N}, where N is the number of `\^{}'s.

\item \texttt{:D\~{}mathematical expression}

If the mathematical expression, or its
modulus, is a non-negative integer, say N, then it is equivalent to \texttt{:D\~{}N}.
Otherwise, it is an invalid vote.

\item \texttt{:D\~{}kick}

Refers to last kick. Can only be used right after a kick, or
a kick followed by a join by the kicked person.

\item \texttt{nick: :D, nick: :D\~{}N, nick: :D\^{}, nick: :D\~{}mathematical expression}

Same as without the \texttt{nick: } prefix, but instead refer to the relevant proposal made by `nick'. \texttt{nick, } can be used instead of \texttt{nick: }.

\end{itemize}

\subsubsection{What counts as a filibuster/proposal?}

\begin{itemize}

\item Filibustering messages \emph{are} proposals, \emph{unless} they're sent by bots to explain something in the previous message (e.g. title bot, automatic translation) \emph{or} they begin with \texttt{nolog:} or \texttt{[nolog]}.
\item Bot messages sent as a response to direct command (e.g. program evaluation, non-automatic translation) \emph{are} proposals.
\item Notices are treated like messages (except they can't be used to vote).
\item Nick changes \emph{are not} proposals, \emph{unless} they are a direct response to something in channel \emph{or} are disruptive.
\item Kicks and mode changes \emph{are} proposals.
\item Joins \emph{are not} proposals, \emph{unless} it is a first join \emph{or} intended to disrupt.
\item Parts and quits \emph{are not} proposals, \emph{unless} intended to disrupt.
\item Special behaviour: valid votes with \texttt{:D:} or \texttt{D:} substituted for \texttt{:D} can \emph{not} act as proposals. These messages
still filibuster.
\end{itemize}

\subsection{Behaving}

\begin{itemize}

\item \url{https://gitlab.com/sortie/mmmm/blob/master/mmmm.txt} (MMMM).
A collection of rules and guidelines that evolved from horrors that won't be mentioned here; currently in helpful form. By new ancient law, MMMM is lawful.

\item Mark NSFW content. Linked NSFW content should be marked, preferably with \texttt{NSFW} or \texttt{[NSFW]}.

\item Do not kick idlers.

Idlers are defined as people whose last activity has been 5 minutes ago (where activity implies messages or nick changes as a response to something in the channel), or who have marked themselves away (e.g. by \texttt{bbl}).

\item Avoid funkicking.
`Funkicking' is where you kick someone just for fun, or for some insignificant reason. Exception to this is if the person you're `funkicking' does not mind the fun kick.

\end{itemize}

\subsection{Additional stuff}

\begin{itemize}
\item In cases where there is disagreement on whether something passed, the lawrememberer's point of view is used.
\item The person who opens the vote on a proposal MUST provide the law to lawrememberer, if requested to do so.
\item Zero-width spaces in votes are to be ignored.
\item It is a good custom to vote on one's own proposal last.
\end{itemize}

\end{document}
